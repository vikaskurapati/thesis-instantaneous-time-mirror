\chapter{Time Reversal}

Time-reversal methods involve mirroring the propagation of waves in time, aiming to converge the resulting wave back to its source.
There are primarily two ways to achive time-reversal(~\parencite{Fink2017}). The first method involves \ac{TRM}s, which rely on 
Cauchy Boundary Conditions. If the wavefield and its normal derivative are known at the entire surface $S$ surrounding the volume
$V$ at all times $t$, the wavefield inside the entire volume can be computed. In this approach, the outgoing wave is initially 
recorded on $S$, then time-reversed and finally emitted from $S$. \\

Alternatively, time-reversal can be achieved with Cauchy Initial Conditions. In this case, the wavefield and its normal derivative
are known inside the entire volume but only for a specific time. By using Lohschmidt daemons, the velocity of each particle can be
instantaneously reversed, resulting in a time-reversed wave. This approach is known as the \ac{ITM}.