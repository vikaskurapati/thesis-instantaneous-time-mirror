\chapter{Time Reversal}

Time-reversal methods involve mirroring the propagation of waves in time, aiming to converge the resulting wave back to its source.
There are primarily two ways to achive time-reversal(~\parencite{Fink2017}). The first method involves \ac{TRM}s, which rely on 
Cauchy Boundary Conditions. If the wavefield and its normal derivative are known at the entire surface $S$ surrounding the volume
$V$ at all times $t$, the wavefield inside the entire volume can be computed. In this approach, the outgoing wave is initially 
recorded on $S$, then time-reversed and finally emitted from $S$. \\

Alternatively, time-reversal can be achieved with Cauchy Initial Conditions. In this case, the wavefield and its normal derivative
are known inside the entire volume but only for a specific time. By using Lohschmidt daemons, the velocity of each particle can be
instantaneously reversed, resulting in a time-reversed wave. This approach is known as the \ac{ITM}.\\

This thesis primarily focuses on the \ac{ITM} approach. In this method, a time-reversed wave is generated through a sudden modification of the wave propagation
properties of the medium(~\parencite{Bacot2016}). The first part of this chapter delves into the theoretical foundation behind the \ac{ITM}, providing a comprehensive
understanding of its principles. In the latter part of the chapter, we present the implementation of \ac{ITM} within the numerical simulation software SeisSol. The
description of the implementation is kept as general as possible, allowing for a broad grasp of the method's application and utility.

\section{Theory of Instantaneous Time Mirrors}

Time-reversal methods are based on the time-reversal invariance of wave equations. They rely on the fact that any wave-field can be completely determined within
a volume by knowing the field(and its normal derivative) on any enclosing surface(~\parencite{Bacot2016}). The Elastic wave equation

\begin{equation}
    \rho \ddot{\mathbf{U}} \left( \mathbf{x}, t\right) - \left( \lambda + 2 \mu \right) \nabla \left(\nabla \cdot \mathbf{U}\left(\mathbf{x}. t\right)\right) 
    + \mu \nabla \times \left(\nabla \times \mathbf{U}\left(\mathbf{x},t\right)\right) = \mathbf{S}\left(\mathbf{x},t\right),
\end{equation}

with the source function $\mathbf{S}\left(\mathbf{x},t\right)$ contains only second order time derivatives. This implies that if $\hat{\mathbf{U}}\left(\mathbf{x},t\right)$
is a solution, then $\hat{\mathbf{U}}\left(\mathbf{x}, -t\right)$ is also a solution, if $\mathbf{S}$ is symmetric in time. \\

The \ac{ITM} approach is closely connected to the Cauchy theorem, which states that the wave field evolution can be deduced from the knowledge of thie wave field
and its derivative at one single time, i.e., the initial conditions. By inducing a sudden modification of the wave propagation properties(such as impedance) in the
medium, a time-reversed wave is generated. This modification, referred to as the \ac{ITM}, does not necessitate the use of antennas or the memory of the entire wave
field(~\parencite{Bacot2016}). ~\parencite{Bacot2016} demonstrated this on water waves with physical and numerical experiments resulting in successful refocusing of the generated waves into their
sources' shapes. This was achieved by introducing a temporal slab in the wave velocities like in figure \ref{fig:deltavelocity}

\begin{figure}
    \centering
    \includegraphics[width=0.6\linewidth]{figures/delta_speed.png}
    \caption{Rectangular profile for the wave velocity.(Figure taken from Figure 1 in ~\parencite[Supplementary Material]{Bacot2016})}
    \label{fig:deltavelocity}
\end{figure}