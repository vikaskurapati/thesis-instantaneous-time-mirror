\chapter{Conclusion and Future Research Outlook}\label{chapter:Conclusion}
In this thesis, we implemented and analysed the wavefield reversal method \ac{ITM} in the numerical simulation software SeisSol. We first presented the basic theory of elastic wave propagation and the numerical \ac{ADER}-\ac{DG} scheme used in SeisSol in chapter \ref{chapter:seismicwaves}.
We then presented the theoretical foundation of the \ac{ITM} method in chapter \ref{chapter:TimeReversal}, the expected solutions for a simple one dimensional case in section \ref{section:ITMAcoustic} and the wavefield reversal for a three dimensional wave in acoustic medium in section \ref{section:3DITMAcoustic}.
The energy behaviour of the \ac{ITM} method was analysed for different phases of the \ac{ITM}.
We then presented the implementation details of the \ac{ITM} method in SeisSol in section \ref{section:Implementation}. The implementation details 
for different wave reversal scenarios were discussed in sections \ref{sec:reflecting_both}, \ref{sec:reflecting_both_constant}, \ref{sec:reflecting_p}, \ref{sec:reflecting_s} and the required modifications for the time step size were discussed in section \ref{sec:time_step_size}. \\

Finally, we presented the results of various studies conducted to analyse the effects of different scenarios on the wavefield reversal in chapter \ref{chapter:Results}.
We first reverse a planar acoustic wave in a homogeneous medium in section \ref{sec:acoustictravelling} and analyse the reversal to verify that the reversed and forward waves have identical speeds as expected from the analytical soutions.
We later reverse spherical waves generated by point sources in WP2-LOH1 case, modified to make the medium homogeneous, in acoustic and elastic media in sections \ref{sec:acousticITM} and \ref{sec:elasticITM}. We verify that the reversal of just one wave is possible while letting 
the other wave stay forward propagating in sections \ref{sec:elasticITMpwave} and \ref{sec:elasticITMswave}. We then verify that our implementation in SeisSol is converging by verifying the numerical results with the analytical solutions developed is increased in section \ref{sec:convergence}. We notice that
the numerical results are converging faster to the analytical solutions as the polynomial order of the \ac{DG} scheme increases. This verifies that the \ac{ITM}
method proposed by ~\parencite{Bacot2016} on waterwaves can be applied analogously on seismic waves to obtain a reversed component retracing them to the source. \\

Currently, the \ac{ITM} parameters are heuristically determined. As part of future research, analytical solutions for the spherical waves produced by point sources may be calculated to select the appropriate parameters for \ac{ITM}. More complex scenarios and wavefield reversal in heterogeneous media can be studied. 
The \ac{ITM} method may be used to study the effects of different wave propagation properties on the wavefield reversal. \ac{ITM} approach and its implementation 
for other scenarios like anisotropic, viscoelastic and poroelastic media can be studied. Noise in the simulation data obtained during the reversal process
could be filtered out to obtain a clearer refocusing pattern and more accurate analysis in the phase shift could be performed.\\