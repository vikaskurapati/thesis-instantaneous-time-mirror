\chapter{Conclusion and Future Research Outlook}\label{chapter:Conclusion}
In this thesis, we implemented and analysed the wavefield reversal method \ac{ITM} in the numerical simulation software SeisSol. We first presented the basic theory of elastic wave propagation and the numerical \ac{ADER}-\ac{DG} scheme used in SeisSol.
We then presented the theoretical foundation of the \ac{ITM} method, the expected solutions for a simple 1D case and the wavefield reversal in 3D in the Acoustic medium.
The energy behaviour of the \ac{ITM} method was analysed for different phases of the \ac{ITM}.
We then presented the implementation details of the \ac{ITM} method in SeisSol. The implementation for different wave reversal scenarios and the required modifications for the time step size were discussed. \\

Finally, we presented the results of the parametric studies conducted to analyse the effects of different scenarios on the wavefield reversal.
We first reverse a planar acoustic wave in a homogeneous medium and analyse the reversal to verify that the reversed and forward waves have identical speeds as expected from the analytical soutions.
We later reverse spherical waves generated by point sources in WP2-LOH1 case modified to make the medium homogeneous, in acoustic and elastic media. We verify that the reversal of just one wave is possible while letting 
the other wave stay forward propagating. We then verify that our implementation in SeisSol is converging by verifying the numerical results with the analytical solutions developed. We notice that
the numerical results are converging faster to the analytical solutions as the order of the \ac{DG} scheme is increased. This verifies that the \ac{ITM}
method proposed by ~\parencite{Bacot2016} on waterwaves can be applied analogously on seismic waves to obtain a reversed component retracing them to the source. \\

We currently choose the \ac{ITM} parameters heuristically. As a part of future research scope, analytical solutions to the spherical waves generated by point sources such that the right parameters are chosen for \ac{ITM} such that 
optimum wave reversal can be obtained. More complex scenarios and wavefield reversal in heterogeneous media can be studied. 
The \ac{ITM} method can be used to study the effects of different wave propagation properties on the wavefield reversal. \ac{ITM} approach and its implementation 
for other scenarios like anisotropic, viscoelastic and poroelastic media can be studied. \\
