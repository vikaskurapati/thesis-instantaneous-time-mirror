% !TeX root = ../main.tex
% Add the above to each chapter to make compiling the PDF easier in some editors.

\chapter{Seismic Waves}\label{chapter:seismicwaves}
In this chapter, we establish the theoretical foundations necessary for this thesis by deriving the fundamental equations for elastic
wave propagation and presenting the equation of motion in velocity-stress formulation. We assume isotropic media and utilize symmetry arguments
throughout the work. The restriction to isotropic media is a commonly employed simplification, as seen in works such as ~\parencite{dumbser1}. \\

The next section of the thesis discusses SeisSol, the numerical tool employed for simulating seismic wave phenomena. SeisSol is a combination
of the Discontinuous Galerkin Finite Element(DG-FE) method and a time integration scheme based on the solution of Arbitrary high-order
Derivative Riemann Problems(ADER), as described in previous works ~\parencite{dumbser1}, ~\parencite{seissol}. This approch, known as 
ADER-DG will be introduced in more detail in the following sections.

\section{Elastic Wave Equation}
An elastic medium is characterized by an undeformed state, in which stresses and strains are zero, to which it will return to in the absence
of outer forces. If the stresses and strains the medium experiences are infinitesimal, the theory of linear elasticity applies. We define
the displacement vector $\mathbf{U}$ describing the shortest distance between the initial and current position of a point. The particle
velocities $u,v$ and $w$ in $x, y, z$ direction, respectively, can then be defined as the temporal derivate of $\mathbf{U}$

\begin{align}
    \begin{split}
    \frac{\partial U_x}{\partial t} = \dot{U}_x = u \\
    \frac{\partial U_y}{\partial t} = \dot{U}_y = v \\
    \frac{\partial U_z}{\partial t} = \dot{U}_z = w, \\
    \end{split}
 \end{align}

where the subscript denotes the coordinate direction and a dot over a variable represents its partial time derivative leading to the following notation:

\begin{align}
    \begin{split}
        \frac{\partial U_i}{\partial t} = \partial_t U_i = \dot{U}_i = V_i,
    \end{split}
    \label{equation1}
\end{align}

where the velocity vector \textbf{V} is introduced. \\

In the context of linear elasticity, the following extension of Hooke's law holds

\begin{align}
    \begin{split}
        \sigma_{ij} = c_{ijkl}\epsilon_{kl},
    \end{split}
\end{align}

with the medium specific constants $c_{ijkl}$ being used to generalize Hooke's law in linear elasticity. The Einstein summation convention
is followed, where an index appearing twice is summed over all possible values. In this context, $\sigma_{ij}$ represents the stress tensor
and $\epsilon_{kl}$ represnets the strain tensor. Hooke's law expresses that stress tensor components are linear combinations of strain tensor
components. For infinitesimally small perturbations, the strain tensor components $\epsilon_{kl}$ are defined as follows

\begin{align}
    \begin{split}
        \epsilon_{kl} = \frac{1}{2}\left(\partial_k U_l + \partial_l U_k \right) ,
    \end{split}
\end{align}

where $\partial_k$ represents the spatial derivative in k-direction and $U_i$ represents the displacement in i-direction.
Previously, we mentioned our focus on the velocity-stress formulation rather than the displacement-stress formulation.
To eliminate the displacements $U_i$, we introduced velocities $V_i$ in equation \ref{equation1}.
Consequently, this leads to the time derivative of the strain tensor

\begin{align}
    \begin{split}
        \dot{\epsilon}_{kl} = \frac{1}{2}\left( \partial_k V_l + \partial_l V_k \right) ,
    \end{split}
\end{align}

expressed in terms of the spatial derivatives of the velocities $V_i$. It is straightforward to calculate the time derivative
of the stress tensor if the constants $c_{ijkl}$ remain constant over time

\begin{align}
    \begin{split}
        \dot{\sigma}_{ij} = c_{ijkl}\dot{\epsilon}_{kl} .
    \end{split}
\end{align}