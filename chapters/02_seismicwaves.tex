% !TeX root = ../main.tex
% Add the above to each chapter to make compiling the PDF easier in some editors.

\chapter{Seismic Waves}\label{chapter:seismicwaves}
In this chapter, we establish the theoretical foundations necessary for this thesis by deriving the fundamental equations for elastic
wave propagation and presenting the equation of motion in velocity-stress formulation. We assume isotropic media and utilize symmetry arguments
throughout the work. The restriction to isotropic media is a commonly employed simplification, as seen in works such as ~\parencite{dumbser1}. \\

The next section of the thesis discusses SeisSol, the numerical tool employed for simulating seismic wave phenomena. SeisSol is a combination
of the \ac{DG-FE} method and a time integration scheme based on the solution of \ac{ADER}, as described in previous works ~\parencite{dumbser1}, ~\parencite{seissol}. This approch, known as 
\ac{ADER}-\ac{DG} will be introduced in more detail in the following sections.

\section{Elastic Wave Equation}\label{section:elasticwaveequation}
An elastic medium is characterized by an undeformed state, in which stresses and strains are zero, to which it will return to in the absence
of outer forces. If the stresses and strains the medium experiences are infinitesimal, the theory of linear elasticity applies. We define
the displacement vector $\mathbf{U}$ describing the shortest distance between the initial and current position of a point. The particle
velocities $u,v$ and $w$ in $x, y, z$ direction, respectively, can then be defined as the temporal derivate of $\mathbf{U}$

\begin{align}
    \begin{split}
    \frac{\partial U_x}{\partial t} = \dot{U}_x = u \\
    \frac{\partial U_y}{\partial t} = \dot{U}_y = v \\
    \frac{\partial U_z}{\partial t} = \dot{U}_z = w, \\
    \end{split}
 \end{align}

where the subscript denotes the coordinate direction and a dot over a variable represents its partial time derivative leading to the following notation:

\begin{align}
    \begin{split}
        \frac{\partial U_i}{\partial t} = \partial_t U_i = \dot{U}_i = V_i,
    \end{split}
    \label{equation1}
\end{align}

where the velocity vector \textbf{V} is introduced. \\

In the context of linear elasticity, the following extension of Hooke's law holds

\begin{align}
    \begin{split}
        \sigma_{ij} = c_{ijkl}\epsilon_{kl},
    \end{split}
    \label{eq:hookeslaw}
\end{align}

with the medium specific constants $c_{ijkl}$ being used to generalize Hooke's law in linear elasticity. The Einstein summation convention
is followed, where an index appearing twice is summed over all possible values. In this context, $\sigma_{ij}$ represents the stress tensor
and $\epsilon_{kl}$ represnets the strain tensor. Hooke's law expresses that stress tensor components are linear combinations of strain tensor
components. For infinitesimally small perturbations, the strain tensor components $\epsilon_{kl}$ are defined as follows

\begin{align}
    \begin{split}
        \epsilon_{kl} = \frac{1}{2}\left(\partial_k U_l + \partial_l U_k \right) ,
    \end{split}
\end{align}

where $\partial_k$ represents the spatial derivative in k-direction and $U_i$ represents the displacement in i-direction.
Previously, we mentioned our focus on the velocity-stress formulation rather than the displacement-stress formulation.
To eliminate the displacements $U_i$, we introduced velocities $V_i$ in equation \ref{equation1}.
Consequently, this leads to the time derivative of the strain tensor

\begin{align}
    \begin{split}
        \dot{\epsilon}_{kl} = \frac{1}{2}\left( \partial_k V_l + \partial_l V_k \right) ,
    \end{split}
\end{align}

expressed in terms of the spatial derivatives of the velocities $V_i$. It is straightforward to calculate the time derivative
of the stress tensor if the constants $c_{ijkl}$ remain constant over time

\begin{align}
    \begin{split}
        \dot{\sigma}_{ij} = c_{ijkl}\dot{\epsilon}_{kl} .
    \end{split}
    \label{eq:stress}
\end{align}

We can apply symmetry considerations to the term $c_{ijkl}$ in equation \ref{eq:hookeslaw} and reduce the number of independent components from 81 to 21(~\parencite{aki2002quantitative} Chapter 2).
This in turn lets us define the tensor in terms of two constants in isotropic media with the general form of

\begin{equation}
    c_{ijkl} = \lambda \delta_{ij}\delta_{kl} + \mu\left(\delta_{ik}\delta_{jl} + \delta_{il}\delta_{jk}\right),
    \label{eq:isotropic}
\end{equation}

where $\delta$ is the Kronecker-delta function and $\lambda, \mu$ are the Lam\'e parameters. \\

We now write a force balance in a volume $V$ within a surface $S$. The momentum within the volume changes at a rate equal
to the forces acting on that volume. Therefore, the forces acting on these comprise a body force and a surface force, which result
from the presence of normal and shear stresses. This can be written mathematically as 

\begin{equation}
    \frac{\partial}{\partial t} \int_V \rho \frac{\partial \mathbf{U}}{\partial t}dV = \int_V \mathbf{f}dV + \oint_S \mathbf{T\left(n\right)}dS,
    \label{eq:forcebalance}
\end{equation}

where $\frac{\partial}{\partial t} \int_V \rho \frac{\partial \mathbf{U}}{\partial t}dV$, is the momentum of the control volume with density $\rho$, and $\mathbf{T}$
is the traction vector which is related to the stress tensor by Cauchy's stress theorem ~\parencite{cauchy-stress-theorem}

\begin{equation}
    T_j\left(n\right) = \sigma_{ij}n_i,
\end{equation}

with the normal vector $n_i$. Replacing $\mathbf{T}$ in \refeq{eq:forcebalance} using the above
,applying Gauss' divergence theorem and reshuffling gives us

\begin{equation}
    \int_V \left(\rho \ddot{U}_i - f_i - \partial_j \sigma_{ij}\right)dV = 0.
\end{equation}

Here we assumed that the stress tensor is symmetric, i.e., $\sigma_{ij}=\sigma_{ji}$. As it is true for any general volume, we can
equate the integrand to zero. This gives us the equation

\begin{equation}
    \rho \ddot{U}_i = f_i + \partial_j \sigma_{ij},
    \label{eq:force}
\end{equation}

expanding it in $j$ and using $V_i = \frac{\partial U_i}{\partial t}$,

\begin{equation}
    \rho \frac{\partial V_i}{\partial t} = f_i + \partial_x \sigma_{xi} + \partial_y \sigma_{yi} + \partial_z \sigma_{zi},
    \label{eq:finalequation}
\end{equation}

The body force $f_i$ will be discussed later in terms of the source terms. \\

Combining equations \ref{eq:stress}, \ref{eq:isotropic}, \ref{eq:finalequation}, we obtain the final 3D 
wave elastic equation for an isotropic medium in velocity-stress formulation

\begin{align}
    \begin{split}
    \frac{\partial \sigma_{xx}}{\partial t} - \left(\lambda  + 2\mu\right)\frac{\partial u}{\partial x} - \lambda \frac{\partial v}{\partial y} - \lambda\frac{\partial w}{\partial z} &= S_1, \\    
    \frac{\partial \sigma_{yy}}{\partial t} - \lambda \frac{\partial u}{\partial x} - \left( \lambda + 2 \mu \right)\frac{\partial v}{\partial y} - \lambda \frac{\partial w}{\partial w} &= S_2, \\
    \frac{\partial \sigma_{zz}}{\partial t} - \lambda \frac{\partial u}{\partial x} - \lambda \frac{\partial v}{\partial y} - \left(\lambda + 2\mu\right)\frac{\partial w}{\partial z} &= S_3, \\
    \frac{\partial \sigma_{xy}}{\partial t} - \mu \left(\frac{\partial v}{\partial x} + \frac{\partial u}{\partial y}\right) &= S_4, \\ 
    \frac{\partial \sigma_{yz}}{\partial t} - \mu \left(\frac{\partial v}{\partial z} + \frac{\partial w}{\partial y}\right) &= S_5, \\
    \frac{\partial \sigma_{xz}}{\partial t} - \mu \left(\frac{\partial u}{\partial z} + \frac{\partial w}{\partial x}\right) &= S_6, \\
    \rho \frac{\partial u}{\partial t} - \frac{\partial \sigma_{xx}}{\partial x} - \frac{\partial \sigma_{xy}}{\partial y} - \frac{\partial \sigma_{xz}}{\partial z} &= \rho S_7, \\
    \rho \frac{\partial v}{\partial t} - \frac{\partial \sigma_{xy}}{\partial x} - \frac{\partial \sigma_{yy}}{\partial y} - \frac{\partial \sigma_{yz}}{\partial z} &= \rho S_8, \\
    \rho \frac{\partial w}{\partial t} - \frac{\partial \sigma_{xz}}{\partial x} - \frac{\partial \sigma_{yz}}{\partial y} - \frac{\partial \sigma_{zz}}{\partial z} &= \rho S_9, \\
\end{split}
\label{eq:setofequations}
\end{align}

where $S_p, p = 1,\dots9,$ are the source terms. \\

Another important term for consideration with this system is the propagation speed of elastic-acoustic waves. In elastic media, we will 
observe two kinds of waves, a primary compression wave and a secondary sheer wave which are called P-waves and S-waves respectively.\\

Inserting the definition of stress $\sigma_{ij}$ from equation \ref{eq:hookeslaw} and $c_{ijkl}$ from equation \ref{eq:isotropic} into equation \ref{eq:force},
we obtain when there are no external sources

\begin{equation}
    \rho \ddot{U}_i = \left(\lambda + \mu \right)\partial_i \partial_j u_j + \mu \partial_j \partial_j u_i.
    \label{eq:einsteinconvention}
\end{equation}

We use a vector relation,

\begin{equation}
    \nabla \times \left(\nabla \times \mathbf{U}\right) = \nabla\left(\nabla \cdot \mathbf{U}\right) - \nabla^2 \mathbf{U}
\end{equation}

to convert equation \ref{eq:einsteinconvention} into a vector equation. Combining these two we get:

\begin{align}
    \begin{split}
        \rho \ddot{\mathbf{U}} &= \left(\lambda + \mu\right)\nabla\left(\nabla\cdot\mathbf{U}\right) + \mu \nabla^2 \mathbf{U} \\
        &= \left(\lambda + \mu\right)\nabla\left(\nabla \cdot \mathbf{U}\right) - \mu \nabla \times \left(\nabla \times \mathbf{U}\right) + \mu \nabla\left(\nabla \cdot \mathbf{U}\right) \\
        \rho \ddot{\mathbf{U}} &= \left(\lambda + 2\mu\right)\nabla\left(\nabla\cdot\mathbf{U}\right) - \mu \nabla\times\left(\nabla\times\mathbf{U}\right). 
    \end{split}
    \label{eq:vectorequation}
\end{align}

~\Parencite[Sec. 2.1]{kaufman} explains the characterisation and the two different wave speeds related to the equation \ref{eq:vectorequation} as

\begin{equation}
    v_p = \sqrt{\frac{\lambda + 2\mu}{\rho}}, v_s = \sqrt{\frac{\mu}{\rho}}.
\end{equation}

These two velocities are the propagation velocities with $v_p$ as the propagation velocity of the P-wave and $v_s$ as the propagation velocity of the S-wave. \\

We get only one propagating wave when we set $\mu = 0$ with $v_p = \sqrt{\frac{\lambda}{\rho}}$ and simplifies the elastic wave propagation to an acoustic case.
We start our time reversal by reversing an acoustic wave(put reference to the section here) and then move on to the elastic case. Further simplifications like 

\begin{equation}
    \sigma_{ij} = 0, \forall i \neq j
\end{equation}

can be derived from the set of equations \ref{eq:setofequations}. All the diagonal elements $\sigma_{ii} = \lambda \partial_j u_j$
are equal to each other which is analogous to the physical pressure in acoustic media. 

\begin{equation}
    p \eqcolon \sigma_{xx} = \sigma_{yy} = \sigma_{zz} = \lambda \partial_iu_i,
\end{equation}

where Einstein summation is used to condense the equation. And the equation of motion condenses to

\begin{equation}
    \rho \frac{\partial^2 \mathbf{U}_i}{\partial t^2} - \partial_i p = 0.
\end{equation}

\section{\ac{ADER}-\ac{DG}}\label{section:ADER-DG}

In this section, we will discuss on the fundamental framework of SeisSol, which is the \ac{ADER}-\ac{DG} method. This section is to provide
a brief understanding about the framework and not to go too much into detail. Interested readers are redirected to ~\parencite{martin}, ~\parencite{Toro2002},~\parencite{dumbser1}, ~\parencite{DUMBSER2005683}, ~\parencite{Toro2001}, ~\parencite{seissol}.

\subsection[DG method]{Discontinuous Galerkin Method}\label{subsection:DG}

We have earlier formulated the elastic wave equation as a system of linear hyperbolic partial differential equataions. ~\parencite{osti_4491151}
first used the \ac{DG-FE} to solve these kind of problems numerically. ~\parencite{Toro2001} formulated the idea of arbitrary high order generalized
Riemann Solvers in a finite volume framework and coined the approach \ac{ADER}. Here, the combination of the generalized Riemann Solvers with \ac{DG}
methods is explained, resulting in \ac{ADER}-\ac{DG} approach (~\parencite{Dumbser2006}). Higher-order piecewise polynomial approximations are used
along with the theory of fluxes across element boundaries from the finite volume methods. The \ac{ADER}-\ac{DG} scheme is an explicit one step scheme.
It advances the solution by a full time step and needs the computation of inter-cell fluxes once per time-step. \\
% Time Derivatives are replaced with space derivatives
% using Cauchy-Kowalevski or Lax-Wendroff procedure to achieve arbitrary high-order accuracy in both space and time.

To solve the 3D elastic wave equation with the \ac{ADER}-\ac{DG} method, we need to transform it into a more compact form of hyperbolic equations

\begin{equation}
    \frac{\partial \mathbf{Q}}{\partial t} + \mathbf{A} \frac{\partial \mathbf{Q}}{\partial x} + 
    \mathbf{B} \frac{\partial \mathbf{Q}}{\partial y} + \mathbf{C} \frac{\partial \mathbf{Q}}{\partial z} = 0,
\end{equation}

where $\mathbf{Q}$ is the state vector which contains the stresses and velocities, i.e.,

\begin{equation}
    \mathbf{Q} = \left(\sigma_{xx}, \sigma_{yy}, \sigma_{zz}, \sigma_{xy}, \sigma_{yz}, \sigma_{xz}, u, v, w \right)^T
\end{equation}

and $\mathbf{A}, \mathbf{B}, \mathbf{C}$ are space-dependent Jacobian matrices. Where

\begin{align}
    \begin{split}
    \mathbf{A} = 
        \begin{bmatrix}
        0 & 0 & 0 & 0 & 0 & 0 & -\left(\lambda + 2\mu\right) & 0 & 0 \\
        0 & 0 & 0 & 0 & 0 & 0 & -\lambda                     & 0 & 0 \\
        0 & 0 & 0 & 0 & 0 & 0 & -\lambda                     & 0 & 0 \\
        0 & 0 & 0 & 0 & 0 & 0 & 0                   & -\mu & 0 \\
        0 & 0 & 0 & 0 & 0 & 0 & 0                   & 0 & 0 \\
        0 & 0 & 0 & 0 & 0 & 0 & 0                   & 0 & -\mu \\
        -\frac{1}{\rho} & 0 & 0 & 0 & 0 & 0 & 0                   & 0 & 0 \\
        0 & 0 & 0 & -\frac{1}{\rho} & 0 & 0 & 0                   & 0 & 0 \\
        0 & 0 & 0 & 0 & 0 & -\frac{1}{\rho} & 0                   & 0 & 0 \\
    \end{bmatrix},
    \end{split}
\end{align}

and matrices $\mathbf{B}, \mathbf{C}$ are essentially similar with non-zero entries rotated around. Interested readers can refer to
~\parencite{dumbser1} for all the three matrices explicitly stated. \\

As we are dealing with unstructured meshes, we divide our computational domain $\Omega \in \mathbb{R}^3$ into tetrahedral elements 
$\tau^{\left(m\right)}$, with unique indices $m\in\mathbb{N}$. $\mathbf{A}, \mathbf{B}, \mathbf{C}$ are assumed to be piecewise constant
inside each tetrahedron. We now introduce a new coordinate system of $\left(\xi, \eta, \zeta \right)$